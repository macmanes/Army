\documentclass[12pt]{article}
\usepackage{wallpaper}
\ULCornerWallPaper{1}{/Users/macmanes/Dropbox/UNH_Letterhead.pdf}
\usepackage{framed, color}
\usepackage{textpos}
\usepackage{hyperref}
\usepackage{setspace}
\usepackage{textcomp}
\definecolor{shadecolor}{gray}{0.85}
\hypersetup{colorlinks=true, urlcolor=blue}
\pdfpagewidth 8.5in
\pdfpageheight 11in
\addtolength{\oddsidemargin}{-1in}
\addtolength{\evensidemargin}{-1in}
\addtolength{\textwidth}{1in}
\onehalfspacing
\usepackage{fancyhdr}
\addtolength{\topmargin}{-.875in}
\addtolength{\textheight}{1in}
\fancyhf{}
\usepackage{fontspec}
\setmainfont{Times New Roman}
\pagestyle{empty}
\usepackage[top=1in, bottom=1in, left=1in, right=1in]{geometry}

\begin{document}

\textsc{ }

\vspace{20 mm}
\begin{textblock}{12.4}(0,.7)
\parindent 0.000000001in
To whom it may concern, 

\parindent 0.2in
\vspace{6 mm}
Please accept my application that is in response to Army Research Office Broad Agency Announcement for Basic and Applied Scientific Research \#W911NF-12-R-0012-02 section 8.2. Every day, soldiers serving in desert environments are faced with the challenges of operating in extreme heat and intense drought. Without sufficient water intake, dehydration may result in cognitive and physical impairment, and in the complete absence of water, death may occur within a matter of hours. Thus, dehydration in desert environments represents a real threat to both soldier performance and safety.  Unlike humans, desert-adapted mammals may live their entire life without water intake. The proposed research aims to understand the physiology and genomics underlying this amazing phenotype, which serves as the critical 1$^{st}$ step towards reducing human's need for water, and the negative effects of dehydration. This proposal leverages an innovative multi-disciplinary approach that will greatly enhance our scientific understanding, as well as the mission of the United States Army. 


\vspace{20mm}
\noindent Matthew MacManes, Ph.D. \hfill \\
University of New Hampshire \\
Assistant Professor of Genome Enabled Biology\\
Department of Molecular, Cellular, \& Biomedical Sciences\\
189 Rudman Hall\\
Durham, NH  03824\\
603-862-4052 \\
matthew.macmanes@unh.edu \\

\end{textblock}












\end{document}