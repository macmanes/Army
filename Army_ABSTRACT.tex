\documentclass[12pt]{article}
\usepackage{framed, color}
\usepackage{textpos}
\usepackage{natbib}
\usepackage{geometry}
\usepackage[hidelinks]{hyperref}
\usepackage{textcomp}
\usepackage{graphicx}
\usepackage{fancybox}
\usepackage{setspace}
\hypersetup{colorlinks=false, urlcolor=blue, citecolor=black}
\usepackage{soul}
\usepackage{geometry}
\usepackage{fancyhdr}
\usepackage{wrapfig}
\usepackage{mdframed}
\usepackage{fontspec}

\renewcommand\refname{Bibliography and References Cited}
\newgeometry{top=1in, bottom=1in, left=1in, right=1in}
\setmainfont{Times New Roman}
\linespread{1.2}
\urlstyle{same}
\setlength{\parindent}{1cm}

\begin{document}

%\setcounter{page}{0}

%\fancyhead[CO]{Matthew D. MacManes | Specific Aims}
%\pagestyle{fancy}
\setcounter{page}{1}
%\pagestyle{empty}
%\raggedright

\begin{center}

 \textbf{PROJECT ABSTRACT} \\

\textsc{}

\end{center}



Environmental stressors faced by soldiers operating in desert environments represent serious threats to their physical and cognitive performance. Heat and aridity, while dangerous to humans, are not harmful to animals adapted to these conditions. This research proposal aims to characterize the physiologic and genomic mechanisms that enable desert-adapted rodents to survive, ultimately leading to strategies aimed at enhancing soldier safety and performance in desert environments. I will accomplish this goal by conducting a series of experiments on captive desert rodents housed in a desert chamber. I will measure multiple physiological parameters and assay the underlying patterns of gene expression, isoform use, and methylation. This project will result in the elucidation of genetic mechanisms that enable survival and lay the foundation for future work aimed at developing interventions, specifically reducing the untoward effects of dehydration on soldier performance.  \\


   































\end{document}
