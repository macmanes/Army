\documentclass[12pt]{article}
\usepackage{framed, color}
\usepackage{textpos}
\usepackage{natbib}
\usepackage{geometry}
\usepackage[hidelinks]{hyperref}
\usepackage{textcomp}
\usepackage{graphicx}
\usepackage{fancybox}
\usepackage{setspace}
\hypersetup{colorlinks=false, urlcolor=blue, citecolor=black}
\usepackage{soul}
\usepackage{geometry}
\usepackage{fancyhdr}
\usepackage{wrapfig}
\usepackage{mdframed}
\usepackage{fontspec}

\renewcommand\refname{Bibliography and References Cited}
\newgeometry{top=1in, bottom=1in, left=1in, right=1in}
\setmainfont{Times New Roman}
\linespread{1.2}
\urlstyle{same}
\setlength{\parindent}{1cm}

\begin{document}

%\setcounter{page}{0}

%\fancyhead[CO]{Matthew D. MacManes | Specific Aims}
%\pagestyle{fancy}
\setcounter{page}{1}
%\pagestyle{empty}
%\raggedright

\begin{center}

 \textbf{PROJECT ABSTRACT} \\

\textsc{}

\end{center}


Every day, soldiers serve in desert environments, forced to endure intense head and aridity. Indeed, these environmental stressors may pose a greater threat than that of enemy combatants. While billions of dollars have been spent on protecting soldiers from bullets, far less attention has been paid to the more insidious threat of heat and dehydration, which may result in cognitive or physical impairment or even death. What if there was a way to significantly enhance the performance and safety of our soldiers by reducing the physiological need for water, particularly in desert environments? While humans and most other mammals are exquisitely sensitive to dehydration, many animals that evolved in deserts are capable of living without ever drinking water. This basic science research proposal aims to understand the genetic and genomic underpinnings of survival without water in a desert-adapted rodent native to the southwest United States. \\


%Environmental stressors faced by soldiers operating in desert environments represent serious threats to their physical and cognitive performance. Heat and aridity, while dangerous to humans, are not harmful to animals adapted to these conditions. This research proposal aims to characterize the physiologic and genomic mechanisms that enable desert-adapted rodents to survive, ultimately leading to strategies aimed at enhancing soldier safety and performance in desert environments. \\

I will accomplish this goal by conducting a series of experiments on captive desert rodents housed in a chamber designed to replicate the intense heat and aridity of desert environments, while allowing me to manipulate other environmental variables such as water availability and diet. I will measure multiple physiological parameters including serum electrolyte and urine concentration, as well as animal behavior, which will allow me to better understand the unique physiology of these animals. To assay the genomic processes that underlie physiology, I use the techniques of computational genomics. Here, I will uncover the genome wide patterns of gene expression and methylation, allowing me to identify specific genes and pathways that will become future targets for therapeutic intervention. \\  

In summary, this project will result in the elucidation of genetic mechanisms that enable survival in desert environments via the use of a unique mammalian model of dehydration. Long term, we strive to leverage our understanding of dehydration resistance in desert rodents to reduce the physical and cognitive deficits associated with human dehydration, thus increasing soldier performance and safety. \\


   































\end{document}
